\documentclass[a4paper]{article}

\usepackage[utf8]{inputenc}
\usepackage[a4paper, margin=3.5cm]{geometry}

\usepackage{graphicx}
\usepackage{url}
\usepackage[spanish,es-tabla]{babel}
\usepackage{fancyhdr}
\usepackage{float}
\usepackage{hyperref}
\usepackage[newfloat]{minted}
\usepackage[nolist]{acronym}
\usepackage{glossaries}
\usepackage[
    type={CC},
    modifier={by-nc-sa},
    version={3.0},
]{doclicense}

\renewcommand{\headrulewidth}{0.6pt}
\renewcommand{\footrulewidth}{0.6pt}

\pagestyle{fancy}
\setlength\headheight{50pt}
\lhead{\includegraphics[height=1.5cm]{Practicas/figures/Practica_1/logos/upm_logo}}
\chead{Métodos generativos\\\vspace{.5em} Práctica Grupal\\\vspace{-.1em}}
\rhead{\includegraphics[height=1.5cm]{Practicas/figures/Practica_1/logos/etsisi_logo}}
\lfoot{\textbf{Tema 2 y 3:} Arquitecturas de metodos generativos}
\cfoot{}
\rfoot{\thepage}

\parskip 1.1ex % paragraph spacing

\newacronym{ae}{AE}{Autoencoder}
\newacronym{vae}{VAE}{Variational Autoencoder}
\newacronym{gan}{GAN}{Generative Adversarial Network}

\begin{document}

\section*{Propuesta de temas}

\begin{itemize}
    \item Unpaired domain-to-domain translation:
    \begin{itemize}
        \item Con \glspl{ae}, \glspl{vae}, \glspl{gan}, Modelos de Diffusion, Transformers...
        \item Se recomienda centrarse en una arquitectura.
    \end{itemize}

    \item Ética y regulación a nivel europeo de modelos generativos.
    \begin{itemize}
        \item Cómo cumplirla y cómo medir cosas como sesgos de género.
    \end{itemize}

    \item Generacion de videos con Diffusion models, GANs,...

    \item Generación de música.
    
\end{itemize}

\section*{Artículos interesantes de referencia}

\begin{itemize}
    \item Ensembling of Convolutional Neural Networks (CNNs) and Vision Transformers (ViTs). Convolutional Vision Transformer (CvT) and LeViT. (Zhang, Belcheva, \& Ermakova, 2025). Y su interpretabilidad (Mienye et al., 2024b) (Djoumessi et al., 2025).

    \item A novel gradient inversion attack framework to investigate privacy vulnerabilities during retinal image-based federated learning (Nielsen et al. 2025).

\end{itemize}
\end{document}
