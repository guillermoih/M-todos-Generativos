\documentclass[10pt,x11names,table]{beamer}

\usetheme[progressbar=frametitle]{metropolis}
\usepackage{appendixnumberbeamer}
\usepackage{xcolor}

\usepackage{polyglossia}
\setmainlanguage{spanish}

\usepackage{listings}

\usepackage{booktabs}
\usepackage[scale=2]{ccicons}

\usepackage{pgfplots}
\usepgfplotslibrary{dateplot}

%ANIMACIONES
\usepackage{animate}
\usepackage{graphicx}
\usepackage[caption=false]{subfig}

\usepackage{xspace}

\newcommand*{\eg}{e.g.\@\xspace}
\newcommand*{\ie}{i.e.\@\xspace}

\let\oldquote\quote
\let\endoldquote\endquote
\renewenvironment{quote}[2][]
  {\if\relax\detokenize{#1}\relax
     \def\quoteauthor{#2}%
   \else
     \def\quoteauthor{#2~---~#1}%
   \fi
   \oldquote}
  {\par\nobreak\smallskip\hfill(\quoteauthor)%
   \endoldquote\addvspace{\bigskipamount}}
   
\usepackage{wrapfig}

\usepackage{subfig}
\usepackage{hyperref}
\usepackage{multicol}

\setbeamertemplate{bibliography item}[text]

\usepackage[font=small,skip=0pt, labelformat=empty]{caption}

\usepackage{dirtytalk}
\usepackage[acronym]{glossaries}
\makeglossaries

\newacronym{acgan}{ACGAN}{Auxiliary Classifier GAN}
\newacronym{ae}{AE}{Autoencoder}
\newacronym{ai}{AI}{Artificial Intelligence}
\newacronym{api}{API}{Application Programming Interface}
\newacronym{bert}{BERT}{Bidirectional Encoder Representations from Transformers}
\newacronym{brief}{BRIEF}{Binary Robust Independent Elementary Features}
\newacronym{brnn}{BRNN}{Bidirectional RNN}
\newacronym{bptt}{BPTT}{Backpropagation Through Time}
\newacronym{cbow}{CBOW}{Continous bag-of-words}
\newacronym{cnn}{CNN}{Convolutional Neural Network}
\newacronym{crnn}{CRNN}{Convolutional Recurrent Neural Network}
\newacronym{ddpm}{DDPM}{Denoising Diffusion Probabilistic Model}
\newacronym{ddim}{DDIM}{Denoising Diffusion Implicit Model}
\newacronym{diffit}{DiffiT}{Diffusion Vision Transformer}
\newacronym{dl}{DL}{Deep Learning}
\newacronym{dnn}{DNN}{Deep Neural Network}
\newacronym{dos}{DoS}{Denial of Service}
\newacronym{drnn}{DRNN}{Deep Recurrent Neural Network}
\newacronym{ecg}{ECG}{Electrocardiogram}
\newacronym{elmo}{ELMo}{Embedding from Language Model}
\newacronym{fast}{FAST}{Features from Accelerated Segment Test}
\newacronym{foss}{FOSS}{Free and open-source software}
\newacronym{gan}{GAN}{Generative Adversarial Network}
\newacronym{glove}{GloVe}{Global Vectors for Word Representation}
\newacronym{gpu}{GPU}{Graphics Processing Unit}
\newacronym{gru}{GRU}{Gated Recurrent Unit}
\newacronym{ilsvrc}{ILSVRC}{ImageNet Large Scale Visual Recognition Challenge}
\newacronym{ldm}{LDM}{Latent Diffusion Model}
\newacronym{lstm}{LSTM}{Long Short-Term Memory}
\newacronym{mape}{MAPE}{Mean Absolute Perentage Error}
\newacronym{ml}{ML}{Machine Learning}
\newacronym{mlp}{MLP}{Multilayer Perceptron}
\newacronym{mse}{MSE}{Mean Squared Error}
\newacronym{ner}{NER}{Named Entity Recognition}
\newacronym{nlg}{NLG}{Natural Language Generation}
\newacronym{nlp}{NLP}{Natural Language Processing}
\newacronym{nlu}{NLU}{Natural Language Understanding}
\newacronym{nn}{NN}{Neural Network}
\newacronym{ocr}{OCR}{Optical Character Recognition}
\newacronym{onnx}{ONNX}{Open Neural Network Exchange}
\newacronym{pmml}{PMML}{Predictive Model Markup Language}
\newacronym{relu}{ReLU}{Rectified Linear Unit}
\newacronym{rest}{REST}{Representational State Transfer}
\newacronym{rnn}{RNN}{Recurrent Neural Network}
\newacronym{sae}{SAE}{Stacked Autoencoder}
\newacronym{sift}{SIFT}{Scale-Invariant Feature Transform}
\newacronym{slam}{SLAM}{Simultaneous Localization and Mapping}
\newacronym{sru}{SRU}{Single Recurrent Unit}
\newacronym{surf}{SURF}{Speeded-Up Robust Features}
\newacronym{svm}{SVM}{Support Vector Machine}
\newacronym{vae}{VAE}{Variational Autoencoder}
\newacronym{vgg}{VGG}{Visual Geometry Group}
\newacronym{vit}{ViT}{Vision Transformer}
\newacronym{wsgi}{WSGI}{Web Server Gateway Interface}
\newacronym{xai}{XAI}{eXplainable Artificial Intelligence}
\newacronym{yolo}{YOLO}{You Only Look Once}
\newacronym{zsl}{ZSL}{Zero-shot Learning}
\subtitle{Métodos Generativos, curso 2025-2026}

\date{\today}
\author{Guillermo Iglesias, guillermo.iglesias@upm.es \newline
Jorge Dueñas Lerín, jorge.duenas.lerin@upm.es  \newline
Edgar Talavera Muñoz, e.talavera@upm.es}

\institute{Escuela Técnica Superior de Ingeniería de Sistemas Informáticos | UPM \newline
\hbox{} \newline \ccbysa \hspace{0.1pt} \ccNonCommercial}

%%%%%%%%%%%%%%%%%%%%%%%%%%%%%%%%%%%%%       
\title{De la Investigación a la Producción: Fine-tuning eficiente y Aplicaciones RAG}

\begin{document}
\maketitle

\begin{frame}{Contenidos}
  \begin{enumerate}
      \item Introducción
      \item Auto-encoders (AEs)
      \item{Auto-encoders Variacionales (VAEs)}
      \item{Generative Adversarial Networks (GANs)}
      \item{Transformers}
      \item{\textbf{Del Transformer Original al Ecosistema Moderno}}
      \item{Diffusion Models}
    \end{enumerate}
\end{frame}

\section{Fine-tuning y Transfer Learning}


\begin{frame}{Full Fine-tuning: la aproximación tradicional}

\textbf{¿Qué es?}
\begin{itemize}
    \item Actualizar \textbf{todos los parámetros} del modelo pre-entrenado
    \item Usar datos etiquetados de la tarea objetivo
    \item Entrenar con learning rate bajo (no olvidar conocimiento previo)
\end{itemize}

\vspace{0.5em}

\begin{columns}
\column{0.5\textwidth}
\textbf{Ventajas}
\begin{itemize}
    \item Máximo rendimiento posible
    \item Simple de implementar
    \item Flexibilidad total
\end{itemize}

\column{0.5\textwidth}
\textbf{Desventajas}
\begin{itemize}
    \item Alto coste computacional
    \item Necesita más datos
    \item Riesgo de overfitting
    \item Catastrophic forgetting
\end{itemize}

\end{columns}

\end{frame}


\begin{frame}{Full Fine-tuning: la aproximación tradicional}

\textbf{Dos problemas principales}

\begin{alertblock}{1. Un modelo completo por tarea}
BERT-base: 110M parámetros -> cada tarea requiere su propia copia del modelo.
\end{alertblock}

\begin{alertblock}{2. Coste de entrenamiento}
Durante el entrenamiento, se actualizan todos los parámetros:
\begin{itemize}
    \item La memoria necesaria crece con el número total de parámetros.
    \item Más gradientes que almacenar → mayor uso de GPU/VRAM.
    \item Entrenamientos más lentos y costosos.
\end{itemize}
\end{alertblock}

\end{frame}



\begin{frame}{Parameter-Efficient Fine-Tuning (PEFT)}

\textbf{Idea central}: ¿Y si solo entrenamos una \textbf{pequeña parte} del modelo?

\begin{block}{Ventajas de PEFT}
\begin{itemize}
    \item Menos memoria durante entrenamiento
    \item Entrenamiento más rápido
    \item Evita catastrophic forgetting
    \item Múltiples tareas con mismo modelo base
\end{itemize}
\end{block}

\textbf{Principales métodos:}
\begin{enumerate}
    \item \textbf{LoRA (Low-Rank Adaptation):} se añaden matrices de bajo rango a los pesos del modelo para ajustarlos sin modificarlos directamente.
    \item \textbf{Adapters:} se insertan pequeñas capas nuevas entre las existentes; solo estas se entrenan.
    \item \textbf{Prompt / Prefix Tuning:} se entrenan vectores de entrada adicionales que guían al modelo, sin tocar sus parámetros internos.
\end{enumerate}

\end{frame}

\begin{frame}{LoRA: Low-Rank Adaptation}

\textbf{Idea clave}:
\begin{itemize}
    \item Los pesos pre-entrenados $W_0$ permanecen \textbf{congelados}
    \item Añadimos matrices de \textbf{bajo rango} $A$ y $B$
    \item Solo entrenamos $A$ y $B$
\end{itemize}

\textbf{Formulación matemática:}
\begin{align*}
h &= W_0 x + \Delta W x \\
\Delta W &= BA
\end{align*}

\vspace{-0.5em}
\begin{itemize}
    \item $x \in \mathbb{R}^k$: vector de entrada.
    \item $h \in \mathbb{R}^d$: salida del modelo (por ejemplo, activación de una capa lineal).
    \item $W_0 \in \mathbb{R}^{d \times k}$: pesos originales (congelados).
    \item $B \in \mathbb{R}^{d \times r}$, $A \in \mathbb{R}^{r \times k}$: matrices entrenables de bajo rango.
    \item $r \ll \min(d, k)$: controla cuántos parámetros nuevos se introducen.
\end{itemize}

\end{frame}


\begin{frame}{LoRA: Low-Rank Adaptation}

\begin{columns}
\column{0.5\textwidth}
\textbf{Hiperparámetros}:
\begin{itemize}
    \item $r$: rank (típicamente 8, 16, 32)
    \item $\alpha$: scaling factor
\end{itemize}
\column{0.5\textwidth}
\begin{figure}
    \centering
    \includegraphics[width=\textwidth]{Slides/figures/llm/peft-lora.png}
    \caption{Esquema de LoRA}
\end{figure}
\end{columns}

\href{Animación}{https://learnopencv.com/fine-tuning-llms-using-peft/}

\begin{exampleblock}{Ejemplo}
\small
GPT-3 (175B parámetros)
LoRA: solo 37M parámetros entrenables (0.02\%)
\end{exampleblock}

\end{frame}





\section{Optimización y Eficiencia}

\begin{frame}{Más técnicas para modelos eficientes}

\textbf{Objetivo:} reducir el tamaño y el coste de los modelos sin perder demasiado rendimiento.


\textbf{1. Knowledge Distillation}
\begin{itemize}
    \item Modelo grande (\textbf{teacher}) entrena a uno más pequeño (\textbf{student}).
    \item El estudiante aprende a imitar las salidas o distribuciones del maestro.
    \item Permite obtener modelos ligeros con un rendimiento similar.
\end{itemize}

\textbf{2. Quantization}
\begin{itemize}
    \item Representar los pesos con menos bits (ej. \texttt{float32} → \texttt{int8}).
    \item Reduce tamaño del modelo y uso de memoria.
    \item Acelera la inferencia con hardware optimizado.
\end{itemize}

\begin{alertblock}{Idea común}
Ambas técnicas buscan modelos \textbf{más pequeños, rápidos y eficientes},  
manteniendo un rendimiento aceptable para tareas reales.
\end{alertblock}

\end{frame}

\begin{frame}{Knowledge Distillation: Teacher-Student}

\begin{figure}
    \centering
    \includegraphics[width=\textwidth]{Slides/figures/llm/kdistil.png}
    \caption{Esquema de Knowledge Distillation}
\end{figure}

\textbf{Objetivo}: Transferir conocimiento de un modelo grande (teacher) a uno pequeño (student)


\end{frame}

\begin{frame}{Knowledge Distillation: Teacher-Student}

\textbf{Proceso}:
\begin{enumerate}
    \item Entrenar modelo grande (teacher)
    \item Usar outputs del teacher como "soft targets"
    \item Entrenar modelo pequeño (student) para imitarlo
\end{enumerate}

\textbf{¿Por qué funciona?}
\begin{itemize}
    \item Las probabilidades suaves contienen más información
    \item El student aprende relaciones entre clases
    \item No solo la clase correcta, sino las similitudes
\end{itemize}



\end{frame}

\begin{frame}{Knowledge Distillation: Función de Pérdida}

\begin{figure}
    \centering
    \includegraphics[width=\textwidth]{Slides/figures/llm/knowledge-distil.png}
    \caption{Knowledge Distillation}
\end{figure}

\textbf{Pérdida combinada}:
\begin{align*}
\mathcal{L} = \alpha \cdot \mathcal{L}_{CE}(\text{hard targets}) + (1-\alpha) \cdot \mathcal{L}_{KL}(\text{soft targets})
\end{align*}

donde:
\begin{itemize}
    \item $\mathcal{L}_{CE}$: cross-entropy con etiquetas reales
    \item $\mathcal{L}_{KL}$: KL-divergence con distribución del teacher
    \item $\alpha$: balance entre ambas pérdidas
\end{itemize}



\end{frame}


\begin{frame}{Knowledge Distillation: Temperatura y caso de éxito}

\begin{exampleblock}{Caso de éxito}
\textbf{DistilBERT} (Hugging Face, 2019):  
\begin{itemize}
    \item Modelo reducido en un 40\% de parámetros respecto a BERT-base.
    \item Velocidad de inferencia un 60\% mayor.
    \item Retiene el 97\% del rendimiento.
\end{itemize}
\end{exampleblock}
\href{https://zilliz.com/learn/distilbert-distilled-version-of-bert}{Fuente de información}
\end{frame}




\begin{frame}{Quantization: Reduciendo Precisión}

\begin{columns}
\column{0.5\textwidth}
\textbf{Idea}: Representar pesos con menos bits

\textbf{Tipos de precisión}:
\begin{itemize}
    \item \textbf{FP32}: 32 bits (full precision)
    \item \textbf{FP16}: 16 bits (half precision)
    \item \textbf{INT8}: 8 bits (integer)
    \item \textbf{INT4}: 4 bits (muy agresivo)
\end{itemize}

\textbf{Beneficios}:
\begin{itemize}
    \item Menos memoria
    \item Inferencia más rápida
    \item Importante para deployment
\end{itemize}

\column{0.5\textwidth}

\begin{figure}
    \centering
    \includegraphics[width=\textwidth]{Slides/figures/llm/quantization.png}
    \caption{Quantization uniforme}
\end{figure}



\end{columns}

\begin{alertblock}{Trade-off}
Menor precisión → posible pérdida de calidad
\end{alertblock}

\end{frame}

\begin{frame}{QLoRA: Quantization + LoRA}

\textbf{Innovación}: Combinar lo mejor de ambos mundos

\textbf{Cómo funciona}:
\begin{enumerate}
    \item Cargar modelo base en \textbf{4-bit} (quantizado)
    \item Añadir \textbf{adaptadores LoRA} en precisión normal
    \item Fine-tune solo los LoRA
    \item Resultado: modelo cuantizado + adaptadores entrenados
\end{enumerate}


\textbf{Ventajas}:
\begin{itemize}
    \item Memoria mínima durante entrenamiento
    \item Permite fine-tune modelos gigantes (65B+) en una GPU
    \item Rendimiento competitivo con full fine-tuning
\end{itemize}

\end{frame}


\section{RAG}

\begin{frame}{El Problema del Conocimiento Estático}

\begin{columns}
\column{0.5\textwidth}
\textbf{Limitaciones de los LLMs}:
\begin{itemize}
    \item Conocimiento "congelado" en el entrenamiento
    \item No saben eventos recientes
    \item No tienen acceso a info privada/corporativa
    \item Pueden "alucinar" información
\end{itemize}

\vspace{0.5em}

\textbf{Ejemplo}:
\begin{itemize}
    \item GPT-4 (cutoff: sep 2021)
    \item Pregunta: "¿Quién ganó el Mundial 2022?"
    \item Respuesta: no lo sabe
\end{itemize}

\column{0.5\textwidth}
\begin{alertblock}{Solución}
\textbf{RAG}: Retrieval-Augmented Generation

\vspace{0.3em}
Darle al modelo acceso a información externa
\end{alertblock}

\vspace{0.5em}

\begin{exampleblock}{Casos de uso}
\begin{itemize}
    \item QA sobre documentación
    \item Chat con PDFs
    \item Búsqueda semántica
    \item Knowledge bases corporativas
\end{itemize}
\end{exampleblock}

\end{columns}

\end{frame}

\begin{frame}{RAG: Arquitectura}

\begin{figure}
    \centering
    \includegraphics[width=0.7\textwidth]{Slides/figures/llm/rag.png}
    \caption{Arquitectura típica de RAG}
\end{figure}

\textbf{Componentes}:
\begin{enumerate}
    \item \textbf{Knowledge Base}: documentos, artículos, base de datos
    \item \textbf{Retriever}: busca documentos relevantes (encoder, sentence-transformers)
    \item \textbf{Generator}: genera respuesta usando contexto (LLM, T5, GPT)
\end{enumerate}

\end{frame}

\begin{frame}{RAG: ¿Cómo Funciona?}

\textbf{Pipeline paso a paso}:

\vspace{0.5em}

\begin{enumerate}
    \item \textbf{Indexación} (offline):
    \begin{itemize}
        \item Procesar documentos
        \item Generar embeddings (ej: sentence-transformers)
        \item Guardar en vector database (FAISS, Pinecone, Weaviate)
    \end{itemize}
    
    \vspace{0.3em}
    
    \item \textbf{Retrieval} (online):
    \begin{itemize}
        \item Usuario hace pregunta
        \item Convertir pregunta a embedding
        \item Buscar documentos más similares (cosine similarity, etc.)
    \end{itemize}
    
    \vspace{0.3em}
    
    \item \textbf{Generation} (online):
    \begin{itemize}
        \item Construir prompt: pregunta + contexto recuperado
        \item LLM genera respuesta basada en contexto
    \end{itemize}
\end{enumerate}

\vspace{0.5em}

\begin{block}{Clave}
El modelo no "memoriza" información, la \textbf{busca y usa} cuando la necesita
\end{block}

\end{frame}

\begin{frame}[fragile]{RAG: Ejemplo de Prompt}

\begin{block}{Prompt construido por RAG}
\small
\begin{verbatim}
Use the following context to answer the question.
Context:
[Documento 1]: Los transformers fueron introducidos 
en 2017 en el paper "Attention is All You Need"...
[Documento 2]: BERT es un modelo encoder-only que 
usa masked language modeling...

Question: ¿Cuándo se inventaron los transformers?

Answer:
\end{verbatim}
\end{block}

\textbf{Ventajas}:
\begin{itemize}
    \item Respuestas basadas en fuentes verificables
    \item Actualización fácil (actualizar base de datos, no reentrenar)
    \item Reduce alucinaciones
\end{itemize}

\end{frame}

\begin{frame}{RAG: Consideraciones Técnicas}

\textbf{Retriever}:
\begin{itemize}
    \item \textbf{Dense retrieval}: embeddings (BERT, sentence-transformers)
    \item \textbf{Information retrieval techniques}: TF-IDF, BM25
    \item \textbf{Hybrid}: combinar ambos
\end{itemize}

\textbf{Chunking Strategy}:
\begin{itemize}
    \item ¿Cuánto texto por chunk? ¿Cuánto overlapping?
\end{itemize}

\textbf{Vector Databases}:
\begin{itemize}
    \item FAISS (Facebook): in-memory, muy rápido
    \item Pinecone, Milvus: managed, escalables
    \item ChromaDB: simple, open-source
\end{itemize}

\textbf{Trade-offs}:
\begin{itemize}
    \item Más documentos recuperados → más contexto pero más coste
    \item Calidad de retrieval crítica para calidad de respuesta
\end{itemize}

\end{frame}


\begin{exercise}
\href{https://colab.research.google.com/}{Práctica: Fine-tuning y Aplicaciones RAG}

\end{exercise}

%%%%%%%%%%%%%%%%%%%%%%%%%%%%

\begin{frame}{Recursos}
\begin{itemize}
    \item PEFT Library: \url{https://github.com/huggingface/peft}
    \item LoRA Paper: \url{https://arxiv.org/abs/2106.09685}
    \item QLoRA Paper: \url{https://arxiv.org/abs/2305.14314}
    \item RLHF Blog (OpenAI): \url{https://openai.com/research/learning-from-human-preferences}
    \item LangChain: \url{https://www.langchain.com/}
    \item LlamaIndex: \url{https://www.llamaindex.ai/}
    \item RAG Survey: \url{https://arxiv.org/abs/2312.10997}
\end{itemize}
\end{frame}

\appendix

\begin{frame}<presentation:0>{License}
    \begin{block}{Tema \texttt{slides-upm}. Puedes obtener sus fuentes en}
        \begin{center}\url{http://gitlab.com/blazaid/slides-upm}\end{center}
    \end{block}
  
    Tanto esta presentación como el tema están licenciados bajo \href{http://creativecommons.org/licenses/by-sa/4.0/}{Creative Commons
  Atribución-CompartirIgual 4.0 Internacional (CC BY-SA 4.0)}.
    \begin{center}\ccbysa\end{center}
\end{frame}

\end{document}
