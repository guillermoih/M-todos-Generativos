\documentclass[10pt,x11names,table]{beamer}

\usetheme[progressbar=frametitle]{metropolis}
\usepackage{appendixnumberbeamer}
\usepackage{xcolor}

\usepackage{polyglossia}
\setmainlanguage{spanish}

\usepackage{listings}

\usepackage{booktabs}
\usepackage[scale=2]{ccicons}

\usepackage{pgfplots}
\usepgfplotslibrary{dateplot}

%ANIMACIONES
\usepackage{animate}
\usepackage{graphicx}
\usepackage[caption=false]{subfig}

\usepackage{xspace}

\newcommand*{\eg}{e.g.\@\xspace}
\newcommand*{\ie}{i.e.\@\xspace}

\let\oldquote\quote
\let\endoldquote\endquote
\renewenvironment{quote}[2][]
  {\if\relax\detokenize{#1}\relax
     \def\quoteauthor{#2}%
   \else
     \def\quoteauthor{#2~---~#1}%
   \fi
   \oldquote}
  {\par\nobreak\smallskip\hfill(\quoteauthor)%
   \endoldquote\addvspace{\bigskipamount}}
   
\usepackage{wrapfig}

\usepackage{subfig}
\usepackage{hyperref}
\usepackage{multicol}

\setbeamertemplate{bibliography item}[text]

\usepackage[font=small,skip=0pt, labelformat=empty]{caption}

\usepackage{dirtytalk}
\usepackage[acronym]{glossaries}
\makeglossaries

\newacronym{acgan}{ACGAN}{Auxiliary Classifier GAN}
\newacronym{ae}{AE}{Autoencoder}
\newacronym{ai}{AI}{Artificial Intelligence}
\newacronym{api}{API}{Application Programming Interface}
\newacronym{bert}{BERT}{Bidirectional Encoder Representations from Transformers}
\newacronym{brief}{BRIEF}{Binary Robust Independent Elementary Features}
\newacronym{brnn}{BRNN}{Bidirectional RNN}
\newacronym{bptt}{BPTT}{Backpropagation Through Time}
\newacronym{cbow}{CBOW}{Continous bag-of-words}
\newacronym{cnn}{CNN}{Convolutional Neural Network}
\newacronym{crnn}{CRNN}{Convolutional Recurrent Neural Network}
\newacronym{ddpm}{DDPM}{Denoising Diffusion Probabilistic Model}
\newacronym{ddim}{DDIM}{Denoising Diffusion Implicit Model}
\newacronym{diffit}{DiffiT}{Diffusion Vision Transformer}
\newacronym{dl}{DL}{Deep Learning}
\newacronym{dnn}{DNN}{Deep Neural Network}
\newacronym{dos}{DoS}{Denial of Service}
\newacronym{drnn}{DRNN}{Deep Recurrent Neural Network}
\newacronym{ecg}{ECG}{Electrocardiogram}
\newacronym{elmo}{ELMo}{Embedding from Language Model}
\newacronym{fast}{FAST}{Features from Accelerated Segment Test}
\newacronym{foss}{FOSS}{Free and open-source software}
\newacronym{gan}{GAN}{Generative Adversarial Network}
\newacronym{glove}{GloVe}{Global Vectors for Word Representation}
\newacronym{gpu}{GPU}{Graphics Processing Unit}
\newacronym{gru}{GRU}{Gated Recurrent Unit}
\newacronym{ilsvrc}{ILSVRC}{ImageNet Large Scale Visual Recognition Challenge}
\newacronym{ldm}{LDM}{Latent Diffusion Model}
\newacronym{lstm}{LSTM}{Long Short-Term Memory}
\newacronym{mape}{MAPE}{Mean Absolute Perentage Error}
\newacronym{ml}{ML}{Machine Learning}
\newacronym{mlp}{MLP}{Multilayer Perceptron}
\newacronym{mse}{MSE}{Mean Squared Error}
\newacronym{ner}{NER}{Named Entity Recognition}
\newacronym{nlg}{NLG}{Natural Language Generation}
\newacronym{nlp}{NLP}{Natural Language Processing}
\newacronym{nlu}{NLU}{Natural Language Understanding}
\newacronym{nn}{NN}{Neural Network}
\newacronym{ocr}{OCR}{Optical Character Recognition}
\newacronym{onnx}{ONNX}{Open Neural Network Exchange}
\newacronym{pmml}{PMML}{Predictive Model Markup Language}
\newacronym{relu}{ReLU}{Rectified Linear Unit}
\newacronym{rest}{REST}{Representational State Transfer}
\newacronym{rnn}{RNN}{Recurrent Neural Network}
\newacronym{sae}{SAE}{Stacked Autoencoder}
\newacronym{sift}{SIFT}{Scale-Invariant Feature Transform}
\newacronym{slam}{SLAM}{Simultaneous Localization and Mapping}
\newacronym{sru}{SRU}{Single Recurrent Unit}
\newacronym{surf}{SURF}{Speeded-Up Robust Features}
\newacronym{svm}{SVM}{Support Vector Machine}
\newacronym{vae}{VAE}{Variational Autoencoder}
\newacronym{vgg}{VGG}{Visual Geometry Group}
\newacronym{vit}{ViT}{Vision Transformer}
\newacronym{wsgi}{WSGI}{Web Server Gateway Interface}
\newacronym{xai}{XAI}{eXplainable Artificial Intelligence}
\newacronym{yolo}{YOLO}{You Only Look Once}
\newacronym{zsl}{ZSL}{Zero-shot Learning}
\subtitle{Métodos Generativos, curso 2025-2026}

\date{\today}
\author{Guillermo Iglesias, guillermo.iglesias@upm.es \newline
Jorge Dueñas Lerín, jorge.duenas.lerin@upm.es  \newline
Edgar Talavera Muñoz, e.talavera@upm.es}

\institute{Escuela Técnica Superior de Ingeniería de Sistemas Informáticos | UPM \newline
\hbox{} \newline \ccbysa \hspace{0.1pt} \ccNonCommercial}

%%%%%%%%%%%%%%%%%%%%%%%%%%%%%%%%%%%%%       
\title{Del Transformer Original al Ecosistema Moderno}

\begin{document}
\maketitle

\begin{frame}{Contenidos}
  \begin{enumerate}
      \item Introducción
      \item Auto-encoders (AEs)
      \item{Auto-encoders Variacionales (VAEs)}
      \item{Generative Adversarial Networks (GANs)}
      \item{Transformers}
      \item{\textbf{Del Transformer Original al Ecosistema Moderno}}
      \item{Diffusion Models}
    \end{enumerate}
\end{frame}

\section{Evolución y Variantes de Transformers}


\begin{frame}{Recap: ¿Por qué funcionó tan bien el transformer original?}

\begin{columns}
\column{0.6\textwidth}
\begin{itemize}
    \item \textbf{Atención}: mecanismo que facilita capturar dependencias a largo plazo sin recurrencia
    \item \textbf{Paralelización}: eliminación de dependencias secuenciales
    \item \textbf{Escalabilidad}: arquitectura capaz de aprovechar grandes cantidades de datos
    \item \textbf{Transferibilidad}: representaciones aprendidas útiles para múltiples tareas
\end{itemize}
\column{0.4\textwidth}
\vspace{0.2em}
\includegraphics[width=\textwidth]{Slides/figures/02_Metodos_Generativos/trans-arch.png}
\end{columns}

Estos principios han dado lugar a tres familias arquitectónicas principales...

\end{frame}

\begin{frame}{Las tres familias arquitectónicas}

\begin{figure}
    \centering
    \includegraphics[width=0.65\textwidth]{Slides/figures/02_Metodos_Generativos/trans-arch5.png}
    \caption{De este diseño original se derivan tres familias: los modelos que entienden (encoder-only), los que generan (decoder-only) y los que traducen de una representación a otra (encoder-decoder).}
\end{figure}
\vspace{-0.4cm}
\begin{itemize}
    \item \textbf{Encoder-only}: solo la parte izquierda (encoders apilados)
    \item \textbf{Decoder-only}: solo la parte derecha (decoders apilados), 
    \begin{itemize}
        \item sin cross-attention pero con masked-attention
    \end{itemize}
    \item \textbf{Encoder-decoder}: la arquitectura completa
\end{itemize}

\end{frame}



% ============================================================
% Fases de Entrenamiento
% ============================================================


\section{Fases entrenamiento, tareas y modelos de ejemplos}

\begin{frame}{Tareas: Upstream vs Downstream}

\textbf{Pretraining} (conocimiento general) → \textbf{Fine-tuning} (tarea específica)

\vspace{0.5em}
\textbf{Upstream Task (Tarea general)}
\begin{itemize}
    \item Objetivo \textbf{inicial} del entrenamiento del modelo base
    \item Grandes volúmenes de datos, principalmente \textbf{no etiquetados}
    \item Aprendizaje \textbf{auto-supervisado o con supervisión débil}
    \item El modelo adquiere patrones lingüísticos y conocimiento general del mundo
\end{itemize}

\vspace{0.3em}
\textbf{Downstream Task (Tarea específica)}
\begin{itemize}
    \item Tarea \textbf{final} o aplicación concreta
    \item Datos más limitados, normalmente \textbf{etiquetados}
    \item Aprendizaje \textbf{supervisado o basado en instrucciones}
    \item El modelo se adapta al dominio o tarea de interés
\end{itemize}

\end{frame}

\note{
En la práctica, el upstream corresponde a la fase de aprendizaje general (donde el modelo aprende a predecir texto, llenar huecos o relacionar tokens), mientras que el downstream se refiere a las adaptaciones o especializaciones: clasificación, resumen, QA, etc.

Entre ambos mundos hay técnicas intermedias como el continued pretraining o el instruction tuning.

Decimos ``principalmente no etiquetados'' y ``normalmente etiquetados'' porque hoy existen variantes que no encajan del todo en esas categorías: por ejemplo, el pretraining puede incluir datos con supervisión débil (como tareas contrastivas o de emparejamiento imagen-texto), y el fine-tuning puede realizarse con datos sintéticos o generados por modelos (instruction tuning, RLHF) en lugar de conjuntos etiquetados manualmente.
}


\begin{frame}{Tareas: otras estrategias}

\textbf{Continued pretraining / Domain Adaptation}
    \begin{itemize}
        \item Corpus de dominio específico sin etiquetar
        \item Mismo \textbf{tipo} de objetivo: aprender más sobre el lenguaje pero con datos especializados
        \item Adapta vocabulario y conocimiento del dominio
        \item Ejemplo: textos médicos, legales, etc.
    \end{itemize}

\textbf{Aprendizaje sobre las tareas en el \textbf{contexto} (In-context learning)}
\begin{itemize} 
    \item Grandes modelos del lenguaje
    \item Grandes ventanas de contexto
    \item Aprender de la tarea sin modificar pesos.
    \item Ejemplo: zero-shot, few-shot, etc.
\end{itemize}

\textbf{Generación Aumentada por Recuperación (\textbf{RAG})}
\begin{itemize}
    \item Vectorización de conocimiento
    \item Inyección en el contexto
\end{itemize}

\end{frame}



\begin{frame}{Taxonomía de Tareas: Auto-supervisadas (sin etiquetas)}

\textbf{Sin etiquetas específicas:}

\begin{description}
    \item[MLM] \textit{Masked Language Modeling}
    \item[-] Predecir tokens enmascarados usando contexto bidireccional. 
    \item[-] \textbf{Ejemplo:} "El gato está \_\_\_ la alfombra" → "sobre".
    
    \item[CLM] \textit{Causal Language Modeling}
    \item[-]Predecir el siguiente token (autoregresivo).  
    \item[-]\textbf{Ejemplo:} "El clima está" → "soleado".
    
    \item[NSP] Next Sentence Prediction — predecir si una oración sigue a otra en el texto original.
    \item[-] \textbf{Ejemplo:} 
    \begin{itemize}
        \item A: "Juan salió de casa."
        \item B: "Luego compró pan."
        \item Etiqueta: Verdadero (sí son consecutivas).
    \end{itemize}
\end{description}

\end{frame}


\begin{frame}{Taxonomía de Tareas: Auto-supervisadas (sin etiquetas)}

\begin{description}

    \item[SOP] \textit{Sentence Order Prediction}
    \item[-] Determinar el orden correcto de oraciones.  
    \item[-] \textbf{Ejemplo:} "Juan salió. Luego compró pan." vs "Luego compró pan. Juan salió."
    
    \item[RTD] \textit{Replaced Token Detection}
    \item[-] Detectar tokens falsos (ELECTRA).  
    \item[-] \textbf{Ejemplo:} "El perro comió pizza" → detectar que "pizza" es improbable.
    
    \item[Span Corruption]  
    \item[-] Predecir secuencias enmascaradas.  
    \item[-] \textbf{Ejemplo:} "El \_\_\_ en el jardín \_\_\_ muy feliz."
    \item[-] seq1: perro corre; seq2: está.
\end{description}

\end{frame}



\begin{frame}{Taxonomía de Tareas: Supervisadas (Fine-tuning)}

\textbf{Con etiquetas específicas:}

\begin{description}
    \item[Text Classification] Categorizar texto completo.  
    \item[-] \textbf{Ejemplo:} "Este producto es excelente"
    \item[-] Sentimiento: positivo.
    
    \item[Token Classification] Etiquetar cada token.  
    \item[-] \textbf{Ejemplo NER:} "Barack Obama nació en Hawái"
    \item[-] [Persona: Barack Obama], [Lugar: Hawái].
    \item[-] \textbf{Ejemplo POS}: Part-of-Speech tagging (sustantivos, verbos, etc.)
     
    \item[Question Answering] Responder preguntas basadas en contexto.  
    \item[-] \textbf{Ejemplo extractivo:}
    \item[-] "Obama fue el presid... nació en Hawái, en la ... ¿Dónde nació Obama?"
    \item[-] Respuesta: "Hawái".
\end{description}

\end{frame}


\begin{frame}{Taxonomía de Tareas: Generación y Transformación}

\begin{description}
    \item[Summarization] Resumir texto largo.  
    \item[-] \textbf{Ejemplo:} Artículo
    \item[-] "Resumen breve".

    
    \item[Paraphrasing] Reformular manteniendo significado.  
    \item[-] \textbf{Ejemplo:} "Hace frío"
    \item[-] "La temperatura es baja".
    
    \item[Dialogue] Responder en conversación.  
    \item[-] \textbf{Ejemplo:} Usuario: "¿Cómo estás?"
    \item[-] Modelo: "Muy bien, gracias".

    \item[Translation] Traducir idiomas.  
    \item[-] \textbf{Ejemplo:} "Hello world"
    \item[-] "Hola mundo".
\end{description}

\end{frame}






\begin{frame}{Tareas: resumen}

\textbf{Upstream Tasks (Pretraining – aprendizaje general)}
\begin{itemize}
  \item \textbf{Auto-supervisadas:} MLM, CLM, NSP, SOP, RTD, Span Corruption
  \item Aprender representaciones y conocimiento general del lenguaje
\end{itemize}

\textbf{Continued Pretraining / Domain Adaptation}
\begin{itemize}
  \item \textbf{Auto-supervisadas:} sobre corpus especializado.
  \item Ejemplo: GPT -> Generamos LegalGPT o MedicalGPT
\end{itemize}

\textbf{Downstream Tasks (Fine-tuning o uso contextual)}
\begin{itemize}
  \item \textbf{Supervisadas:} Text / Token Classification, QA, Summarization, Paraphrasing, Dialogue, Translation
  \item \textbf{Sin actualización de pesos:} In-context Learning (zero-shot, few-shot)
\end{itemize}

\textbf{Otras Estrategias:}
\begin{itemize}
    \item RAG (Retrieval-Augmented Generation)
\end{itemize}

\end{frame}


\begin{frame}{Tareas: resumen}


\begin{block}{Analogía educativa}

Pensemos en el modelo como una persona que aprende. \\[0.5em]

\textbf{Pretraining:} la etapa escolar. Aprende a leer, escribir y el uso del lenguaje.  
No hay tareas concretas, solo comprensión general del lenguaje y del mundo. \\[0.5em]

\textbf{Domain adaptation:} la universidad. Se especializa en un campo (medicina, derecho, programación)
Aprende con textos del dominio, pero sigue sin etiquetas. \\[0.5em]

\textbf{Fine-tuning:} el trabajo. Aprende una tarea concreta con ejemplos reales.  
Ya no estudia teoría: aplica lo aprendido para resolver problemas específicos. \textbf{Ejemplos:} un médico que aprende a diagnosticar, o un abogado que redacta demandas

\end{block}
\end{frame}


\section{Ejemplos de modelos}


\begin{frame}{Familia 1: Encoder-only — BERT}

\textbf{BERT}: \textit{Bidirectional Encoder Representations from Transformers}

\textbf{Características principales}
\begin{itemize}
    \item \textbf{Arquitectura:} sólo usa \textit{encoders} del Transformer
    \item \textbf{Bidireccionalidad:} cada token ve todo el contexto (izquierda y derecha)
    \item \textbf{Objetivo de entrenamiento:} 
    \textit{Masked Language Modeling (MLM)} y \textit{Next Sentence Prediction (NSP)}
    \item Produce \textbf{representaciones contextuales} ricas
\end{itemize}

\centering
\includegraphics[width=0.75\textwidth]{Slides/figures/llm/bert.png}

\end{frame}


\begin{frame}{Entrenamiento de BERT (Pre-training)}

\textbf{Objetivo:} aprender representaciones lingüísticas generales a partir de grandes corpus sin supervisión.

\textbf{Datos de entrenamiento}: \textit{BooksCorpus} y \textit{Wikipedia} en inglés

\textbf{Tareas de pre-entrenamiento}
\begin{itemize}
    \item Ambas cabezas (MLM y NSP) se entrenan en paralelo $L = L_{\text{MLM}} + L_{\text{NSP}}$

    \item \textbf{Masked Language Modeling (MLM)}  
    \begin{itemize}
        \item Se selecciona aleatoriamente el \textbf{15\% de los tokens} para intentar predecirlos.
        \item De esos tokens seleccionados:
        \begin{itemize}
            \item 80\% se reemplazan por \texttt{[MASK]}
            \item 10\% se reemplazan por un token aleatorio
            \item 10\% se dejan sin modificar (eliminar dependencia de \texttt{[MASK]})
            % Oye, aquí tienes una oración normal. Pero aunque la palabra esté visible, quiero que igualmente trates de predecirla.
        \end{itemize}
    \end{itemize}
    \item \textbf{Next Sentence Prediction (NSP)}  
    \begin{itemize}
        \item 50\% de los pares son oraciones consecutivas (\textit{IsNext})
        \item 50\% son oraciones no relacionadas (\textit{NotNext})
    \end{itemize}
\end{itemize}

\end{frame}



\begin{frame}{Representaciones y entrenamiento de BERT}

\textbf{El token [CLS]} Se añade al inicio de la secuencia
\begin{itemize}
    \item Su vector final \textbf{resume el significado global} del texto.
    \item Se usa para tareas de clasificación (\textit{NSP}, análisis de sentimiento, etc.).
\end{itemize}

\textbf{Embeddings por token}
\begin{itemize}
    \item Cada token obtiene una representación contextual completa.
    \item Útiles para tareas de secuencia (NER, QA, POS tagging).
\end{itemize}


\begin{block}{Ejemplo}
\vspace{0.2cm}
\small
\texttt{Input: "El [MASK] duerme. Está [MASK]."}\\
\texttt{ Objetivo MLM: predecir "gato", "cansado"}\\[0.3em]
\texttt{ Objetivo NSP: IsNext}\\
\vspace{0.2cm}
\texttt{Input: "El [MASK] duerme. Los trenes son [MASK]."}\\
\texttt{ Objetivo MLM: predecir "gato", "rápidos"}\\[0.3em]
\texttt{ Objetivo NSP: NotNext}\\
\end{block}

\end{frame}



\begin{frame}{Familia 2: Decoder-only — GPT-2}

\textbf{GPT-2}: \textit{Generative Pre-trained Transformer 2}

\begin{columns}[T]
\column{0.55\textwidth}

\textbf{Características principales}
\begin{itemize}
    \item Usa solo la parte \textbf{decoder} del Transformer.
    \item Entrenamiento \textbf{autoregresivo}: predice el siguiente token dado el contexto previo.
    \item Objetivo de entrenamiento: \textbf{Causal Language Modeling (CLM)}.
    \item Atención \textbf{unidireccional}: cada token solo ve los anteriores.
\end{itemize}

\column{0.45\textwidth}
\centering
\includegraphics[width=1\textwidth]{Slides/figures/llm/gpt2.png}
\href{https://medium.com/@vipul.koti333/from-theory-to-code-step-by-step-implementation-and-code-breakdown-of-gpt-2-model-7bde8d5cecda}{Link a fuente de imagen} \\
¡Ojo! Masked MHA
\end{columns}

\vspace{0.5cm}

\textbf{GPT-2}: \textit{¡Ojo! se entrena en paralelo. En inferencia se va token a token.}

\end{frame}



\begin{frame}{Generación con GPT-2}

\textbf{Objetivo:} generar texto coherente y fluido, un token a la vez.

\textbf{Durante la generación}
\begin{itemize}
    \item El modelo calcula la probabilidad de cada token posible:
    \[
    P(x_t | x_{<t})
    \]
    \item Se elige el siguiente token según una estrategia de muestreo.
    \item El proceso continúa hasta alcanzar un token de parada o el límite de longitud.
\end{itemize}

\textbf{Estrategias de muestreo}
\begin{itemize}
    \item \textbf{Greedy}: elige siempre el token más probable (poco creativo).
    \item \textbf{Sampling}: elige aleatoriamente según la distribución de probabilidad.
    \item \textbf{Top-$k$ / Nucleus (Top-$p$)}: limita el muestreo a los tokens más probables.
\end{itemize}

\end{frame}


\begin{frame}{Generación con GPT-2}

\begin{block}{Ejemplo de generación}
\small
\texttt{Input: "El gato duerme en"} $\rightarrow$ \texttt{"el sofá."}\\[0.3em]
\texttt{Input: "La IA transformará"} $\rightarrow$ \texttt{"la educación en el futuro."}
\end{block}

\vspace{0.4em}

\begin{block}{Temperatura y creatividad}
\small
Controla la aleatoriedad en el muestreo:\\
\[
P_i' = \frac{\exp(\log P_i / T)}{\sum_j \exp(\log P_j / T)}
\]
\textbf{Temperatura baja} $(T<1)$ → texto predecible\\
\textbf{Temperatura alta} $(T>1)$ → texto más variado
\end{block}
\textbf{Casos de uso}:
\begin{itemize}
    \item Resumen de textos
    \item Paráfrasis
    \item Question Answering generativo
    \item Traducción automática (También aunque se usan más enc-dec) 
\end{itemize}

\end{frame}




\begin{frame}{Familia 3: Encoder–Decoder - T5}

\textbf{T5}: \textit{Text-To-Text Transfer Transformer}

\textbf{Características principales}:
\begin{itemize}
    \item \textbf{Arquitectura completa}: encoder bidireccional + decoder autoregresivo.
    \item Entrenamiento basado en \textbf{denoising}: el modelo aprende a reconstruir texto corrupto.
    \item Formulación unificada: todo se expresa como una tarea \textbf{text-to-text}.
    \item Máxima \textbf{flexibilidad} para tareas de tipo seq2seq.
\end{itemize}

\end{frame}



\begin{frame}{Familia 3: Encoder–Decoder - T5}

\centering
\includegraphics[width=1\textwidth]{Slides/figures/llm/t5.png}


\textbf{Casos de uso}:
\begin{itemize}
    \item Traducción automática
    \item Resumen de textos
    \item Paráfrasis y reformulación
    \item Question Answering generativo
\end{itemize}

% Corpus of Linguistic Acceptability
% Semantic Textual Similarity Benchmark

\end{frame}



\begin{frame}{Comparación de las tres familias}

\begin{table}[]
\centering
\small
\begin{tabular}{|l|l|l|l|}
\hline
\textbf{Familia} & \textbf{Atención} & \textbf{Objetivo} & \textbf{Uso principal} \\ \hline
Encoder-only & Bidireccional & MLM & Comprensión \\ \hline
Decoder-only & Causal & CLM & Generación \\ \hline
Encoder-decoder & Ambas & Seq2seq & Transformación \\ \hline
\end{tabular}
\end{table}

\vspace{1em}

\begin{block}{Regla general}
\begin{itemize}
    \item ¿Necesitas \textbf{entender}? → Encoder-only
    \item ¿Necesitas \textbf{generar}? → Decoder-only
    \item ¿Necesitas \textbf{transformar}? → Encoder-decoder
\end{itemize}
\end{block}

\end{frame}


\section{Conceptos Clave del Ecosistema}



\begin{frame}{Paradigmas de Aprendizaje}

\textbf{1. Transfer Learning: Pretraining + Fine-tuning}
\begin{itemize}
    \item Entrenamiento en dos fases
    \item Modifica los parámetros del modelo
    \item Requiere dataset de la tarea objetivo
    \item Ejemplo: BERT pre-entrenado -> fine-tuned para clasificación de sentimiento
    \item Ejemplo: BERT pre-entrenado -> Domain adaptation "legal" -> fine-tuned para clasificación de posibles veredictos
\end{itemize}

\vspace{0.5em}

\textbf{2. In-Context Learning (emergente en modelos grandes)}
\begin{itemize}
    \item Solo disponible en modelos suficientemente grandes
    \item Ventana grande, conocimiento amplio
    \item \textbf{No modifica} los parámetros del modelo
    \item Aprende de los ejemplos en el prompt
\end{itemize}

\end{frame}

\begin{frame}{In-Context Learning: Zero-shot y Few-shot}

\textbf{Zero-shot}: Sin ejemplos
\begin{block}{}
\small
\texttt{Classify the sentiment of this text: "I love this product!"}

\texttt{Answer: Positive}
\end{block}

\vspace{0.5em}

\textbf{Few-shot}: Con pocos ejemplos (2-5 típicamente)
\begin{block}{}
\small
\texttt{Classify sentiment:}

\texttt{Text: "Great service!" → Positive}

\texttt{Text: "Terrible experience." → Negative}

\texttt{Text: "I love this!" → ?}
\end{block}

\vspace{0.5em}

\begin{alertblock}{Diferencia clave}
In-context learning NO actualiza parámetros, solo usa el contexto del prompt.
\end{alertblock}

\end{frame}

\begin{frame}{Prompt Engineering}

\begin{itemize}
    \item \textbf{Disciplina} que surgió con el auge de los LLMs
    \item Técnicas principales:
    \begin{itemize}
        \item \textbf{Instruction prompting}: instrucciones claras y específicas
        \item \textbf{Few-shot prompting}: proporcionar ejemplos
        \item \textbf{Chain-of-thought}: pedir razonamiento paso a paso
        \item \textbf{Role prompting}: asignar un rol al modelo
    \end{itemize}
    \item Arte y ciencia de diseñar instrucciones efectivas
    \begin{itemize}
        \item CoT -> Modelos razonadores
    \end{itemize}
    \item El prompt adecuado puede marcar la diferencia entre éxito y fracaso
\end{itemize}



\end{frame}





\begin{frame}{Chain of Thought (CoT) y Modelos razonadores}

\small
\textbf{Chain of Thought:} Pedir al modelo que piense y haga explícito el razonamiento intermedio de una respuesta/tarea.  

\begin{itemize}
    \item \textbf{Resultado:} En lugar de dar directamente la respuesta final, el modelo genera una secuencia de pasos lógicos antes de llegar a la solución.
    \item \textbf{Ejemplo:} Para resolver un problema matemático, el modelo escribe los cálculos y justificaciones antes de dar el resultado.
    \item \textbf{Objetivo:} Mejorar la precisión en tareas complejas (matemáticas, lógica, planificación) al permitir que el modelo “razone” de forma estructurada.
\end{itemize}

\textbf{Modelos razonadores:} Modelos diseñados y optimizados específicamente para razonar de forma más profunda y fiable. \textbf{Técnicas empleadas:}
\begin{itemize}
    \item \textbf{Entrenamiento con CoT:} aprenden a generar pasos intermedios.
    \item \textbf{Self-Consistency:} generan varias cadenas de razonamiento y eligen la más coherente.
    \item \textbf{Verificación interna:} el modelo revisa sus propios pasos antes de responder.
\end{itemize}

\end{frame}


\section{Herramientas: Hugging Face}

\begin{frame}{Hugging Face Hub: El GitHub de los modelos}

\begin{columns}
\column{0.5\textwidth}
\textbf{¿Qué es Hugging Face Hub?}
\begin{itemize}
    \item Plataforma centralizada para compartir:
    \begin{itemize}
        \item Modelos pre-entrenados (180,000+)
        \item Datasets
        \item Spaces (demos interactivas)
    \end{itemize}
    \item \textbf{Model cards}: documentación estandarizada
    \begin{itemize}
        \item Arquitectura y tamaño
        \item Dataset de entrenamiento
        \item Métricas de rendimiento
        \item Limitaciones conocidas
        \item Consideraciones éticas
    \end{itemize}
\end{itemize}

\column{0.5\textwidth}
\textbf{Biblioteca transformers}
\begin{itemize}
    \item Abstracción unificada
    \item Mismo código para todos los modelos
    \item API simple:
\end{itemize}

\begin{block}{}
\tiny
\texttt{from transformers import pipeline}

\texttt{classifier = pipeline("sentiment-analysis")}

\texttt{result = classifier("I love this!")}
\end{block}

\end{columns}

\end{frame}






\section{Notebook aplicando conceptos}



\begin{exercise}
\href{https://colab.research.google.com/}{Práctica: Explorando herramientas y arquitecturas}

\end{exercise}

%%%%%%%%%%%%%%%%%%%%%%%%%%%%

\begin{frame}{Recursos}
\begin{itemize}
    \item Hugging Face Hub: \url{https://huggingface.co/models}
    \item Transformers Documentation: \url{https://huggingface.co/docs/transformers}
    \item Ollama: \url{https://ollama.ai/}
    \item The Illustrated Transformer: \url{http://jalammar.github.io/illustrated-transformer/}
    \item Prompt Engineering Guide: \url{https://www.promptingguide.ai/}
\end{itemize}
\end{frame}

\appendix

\begin{frame}<presentation:0>{License}
    \begin{block}{Tema \texttt{slides-upm}. Puedes obtener sus fuentes en}
        \begin{center}\url{http://gitlab.com/blazaid/slides-upm}\end{center}
    \end{block}
  
    Tanto esta presentación como el tema están licenciados bajo \href{http://creativecommons.org/licenses/by-sa/4.0/}{Creative Commons
  Atribución-CompartirIgual 4.0 Internacional (CC BY-SA 4.0)}.
    \begin{center}\ccbysa\end{center}
\end{frame}

\end{document}
