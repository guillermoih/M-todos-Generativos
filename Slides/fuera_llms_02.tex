\documentclass[10pt,x11names,table]{beamer}

\usetheme[progressbar=frametitle]{metropolis}
\usepackage{appendixnumberbeamer}
\usepackage{xcolor}

\usepackage{polyglossia}
\setmainlanguage{spanish}

\usepackage{listings}

\usepackage{booktabs}
\usepackage[scale=2]{ccicons}

\usepackage{pgfplots}
\usepgfplotslibrary{dateplot}

%ANIMACIONES
\usepackage{animate}
\usepackage{graphicx}
\usepackage[caption=false]{subfig}

\usepackage{xspace}

\newcommand*{\eg}{e.g.\@\xspace}
\newcommand*{\ie}{i.e.\@\xspace}

\let\oldquote\quote
\let\endoldquote\endquote
\renewenvironment{quote}[2][]
  {\if\relax\detokenize{#1}\relax
     \def\quoteauthor{#2}%
   \else
     \def\quoteauthor{#2~---~#1}%
   \fi
   \oldquote}
  {\par\nobreak\smallskip\hfill(\quoteauthor)%
   \endoldquote\addvspace{\bigskipamount}}
   
\usepackage{wrapfig}

\usepackage{subfig}
\usepackage{hyperref}
\usepackage{multicol}

\setbeamertemplate{bibliography item}[text]

\usepackage[font=small,skip=0pt, labelformat=empty]{caption}

\usepackage{dirtytalk}
\usepackage[acronym]{glossaries}
\makeglossaries

\newacronym{acgan}{ACGAN}{Auxiliary Classifier GAN}
\newacronym{ae}{AE}{Autoencoder}
\newacronym{ai}{AI}{Artificial Intelligence}
\newacronym{api}{API}{Application Programming Interface}
\newacronym{bert}{BERT}{Bidirectional Encoder Representations from Transformers}
\newacronym{brief}{BRIEF}{Binary Robust Independent Elementary Features}
\newacronym{brnn}{BRNN}{Bidirectional RNN}
\newacronym{bptt}{BPTT}{Backpropagation Through Time}
\newacronym{cbow}{CBOW}{Continous bag-of-words}
\newacronym{cnn}{CNN}{Convolutional Neural Network}
\newacronym{crnn}{CRNN}{Convolutional Recurrent Neural Network}
\newacronym{ddpm}{DDPM}{Denoising Diffusion Probabilistic Model}
\newacronym{ddim}{DDIM}{Denoising Diffusion Implicit Model}
\newacronym{diffit}{DiffiT}{Diffusion Vision Transformer}
\newacronym{dl}{DL}{Deep Learning}
\newacronym{dnn}{DNN}{Deep Neural Network}
\newacronym{dos}{DoS}{Denial of Service}
\newacronym{drnn}{DRNN}{Deep Recurrent Neural Network}
\newacronym{ecg}{ECG}{Electrocardiogram}
\newacronym{elmo}{ELMo}{Embedding from Language Model}
\newacronym{fast}{FAST}{Features from Accelerated Segment Test}
\newacronym{foss}{FOSS}{Free and open-source software}
\newacronym{gan}{GAN}{Generative Adversarial Network}
\newacronym{glove}{GloVe}{Global Vectors for Word Representation}
\newacronym{gpu}{GPU}{Graphics Processing Unit}
\newacronym{gru}{GRU}{Gated Recurrent Unit}
\newacronym{ilsvrc}{ILSVRC}{ImageNet Large Scale Visual Recognition Challenge}
\newacronym{ldm}{LDM}{Latent Diffusion Model}
\newacronym{lstm}{LSTM}{Long Short-Term Memory}
\newacronym{mape}{MAPE}{Mean Absolute Perentage Error}
\newacronym{ml}{ML}{Machine Learning}
\newacronym{mlp}{MLP}{Multilayer Perceptron}
\newacronym{mse}{MSE}{Mean Squared Error}
\newacronym{ner}{NER}{Named Entity Recognition}
\newacronym{nlg}{NLG}{Natural Language Generation}
\newacronym{nlp}{NLP}{Natural Language Processing}
\newacronym{nlu}{NLU}{Natural Language Understanding}
\newacronym{nn}{NN}{Neural Network}
\newacronym{ocr}{OCR}{Optical Character Recognition}
\newacronym{onnx}{ONNX}{Open Neural Network Exchange}
\newacronym{pmml}{PMML}{Predictive Model Markup Language}
\newacronym{relu}{ReLU}{Rectified Linear Unit}
\newacronym{rest}{REST}{Representational State Transfer}
\newacronym{rnn}{RNN}{Recurrent Neural Network}
\newacronym{sae}{SAE}{Stacked Autoencoder}
\newacronym{sift}{SIFT}{Scale-Invariant Feature Transform}
\newacronym{slam}{SLAM}{Simultaneous Localization and Mapping}
\newacronym{sru}{SRU}{Single Recurrent Unit}
\newacronym{surf}{SURF}{Speeded-Up Robust Features}
\newacronym{svm}{SVM}{Support Vector Machine}
\newacronym{vae}{VAE}{Variational Autoencoder}
\newacronym{vgg}{VGG}{Visual Geometry Group}
\newacronym{vit}{ViT}{Vision Transformer}
\newacronym{wsgi}{WSGI}{Web Server Gateway Interface}
\newacronym{xai}{XAI}{eXplainable Artificial Intelligence}
\newacronym{yolo}{YOLO}{You Only Look Once}
\newacronym{zsl}{ZSL}{Zero-shot Learning}
\subtitle{Métodos Generativos, curso 2025-2026}

\date{\today}
\author{Guillermo Iglesias, guillermo.iglesias@upm.es \newline
Jorge Dueñas Lerín, jorge.duenas.lerin@upm.es  \newline
Edgar Talavera Muñoz, e.talavera@upm.es}

\institute{Escuela Técnica Superior de Ingeniería de Sistemas Informáticos | UPM \newline
\hbox{} \newline \ccbysa \hspace{0.1pt} \ccNonCommercial}

%%%%%%%%%%%%%%%%%%%%%%%%%%%%%%%%%%%%%       
\title{De la Investigación a la Producción: Fine-tuning y Optimización}

\begin{document}
\maketitle

\begin{frame}{Contenidos}
  \begin{enumerate}
      \item Introducción
      \item Auto-encoders (AEs)
      \item{Auto-encoders Variacionales (VAEs)}
      \item{Generative Adversarial Networks (GANs)}
      \item{Transformers}
      \item{\textbf{Del Transformer Original al Ecosistema Moderno}}
      \item{Diffusion Models}
    \end{enumerate}
\end{frame}

\section{Fine-tuning y Transfer Learning}

\begin{frame}{Recordatorio: Transfer Learning}

\begin{figure}
    \centering
    \includegraphics[width=0.8\textwidth]{Slides/figures/02_Metodos_Generativos/trans-arch5.png}
\end{figure}

\textbf{Transfer Learning} = Pretraining + Fine-tuning

\begin{itemize}
    \item \textbf{Pretraining}: aprender representaciones generales (upstream task)
    \item \textbf{Fine-tuning}: adaptar a tarea específica (downstream task)
\end{itemize}

Pero... ¿cómo hacer fine-tuning de forma eficiente?

\end{frame}

\begin{frame}{Full Fine-tuning: La Aproximación Tradicional}

\textbf{¿Qué es?}
\begin{itemize}
    \item Actualizar \textbf{todos los parámetros} del modelo pre-entrenado
    \item Usar datos etiquetados de la tarea objetivo
    \item Entrenar con learning rate bajo (no olvidar conocimiento previo)
\end{itemize}

\vspace{0.5em}

\begin{columns}
\column{0.5\textwidth}
\textbf{Ventajas} ✓
\begin{itemize}
    \item Máximo rendimiento posible
    \item Simple de implementar
    \item Flexibilidad total
\end{itemize}

\column{0.5\textwidth}
\textbf{Desventajas} ✗
\begin{itemize}
    \item Alto coste computacional
    \item Necesita más datos
    \item Riesgo de overfitting
    \item Catastrophic forgetting
\end{itemize}

\end{columns}

\vspace{1em}

\begin{alertblock}{Problema principal}
BERT-base: 110M parámetros → guardar modelo completo para cada tarea
\end{alertblock}

\end{frame}

\begin{frame}{Parameter-Efficient Fine-Tuning (PEFT)}

\textbf{Idea central}: ¿Y si solo entrenamos una \textbf{pequeña parte} del modelo?

\vspace{0.5em}

\begin{block}{Ventajas de PEFT}
\begin{itemize}
    \item Menos memoria durante entrenamiento
    \item Menos datos necesarios
    \item Entrenamiento más rápido
    \item Evita catastrophic forgetting
    \item Múltiples tareas con mismo modelo base
\end{itemize}
\end{block}

\vspace{0.5em}

\textbf{Principales técnicas}:
\begin{enumerate}
    \item \textbf{LoRA} (Low-Rank Adaptation)
    \item \textbf{Adapters}
    \item \textbf{Prompt Tuning / Prefix Tuning}
\end{enumerate}

\end{frame}

\begin{frame}{LoRA: Low-Rank Adaptation}

\begin{columns}
\column{0.6\textwidth}
\textbf{Idea clave}:
\begin{itemize}
    \item Los pesos pre-entrenados permanecen \textbf{congelados}
    \item Añadimos matrices de \textbf{bajo rango} $A$ y $B$
    \item Solo entrenamos $A$ y $B$
\end{itemize}

\vspace{0.5em}

\textbf{Matemáticamente}:
\begin{align*}
h &= W_0 x + \Delta W x \\
\Delta W &= BA
\end{align*}

donde $B \in \mathbb{R}^{d \times r}$, $A \in \mathbb{R}^{r \times k}$, y $r \ll d$

\vspace{0.5em}

\textbf{Hiperparámetros}:
\begin{itemize}
    \item $r$: rank (típicamente 8, 16, 32)
    \item $\alpha$: scaling factor
\end{itemize}

\column{0.4\textwidth}
\begin{figure}
    \centering
    %\includegraphics[width=\textwidth]{Slides/figures/02_Metodos_Generativos/lora-diagram.png}
    \caption{Esquema de LoRA}
\end{figure}

\vspace{0.5em}

\begin{exampleblock}{Ejemplo}
\small
GPT-3 (175B parámetros)

LoRA: solo 37M parámetros entrenables (0.02\%)
\end{exampleblock}

\end{columns}

\end{frame}

\begin{frame}{Adapters: Módulos Insertables}

\begin{columns}
\column{0.6\textwidth}
\textbf{Arquitectura}:
\begin{itemize}
    \item Insertar \textbf{módulos pequeños} entre capas del transformer
    \item Estructura típica: down-projection → activation → up-projection
    \item Solo entrenar los adapters
\end{itemize}

\vspace{0.5em}

\textbf{Características}:
\begin{itemize}
    \item Bottleneck architecture (reducir dimensión)
    \item Residual connection
    \item Típicamente $<5\%$ de parámetros del modelo
\end{itemize}

\vspace{0.5em}

\textbf{Ventajas}:
\begin{itemize}
    \item Modular: fácil añadir/quitar
    \item Un modelo base, múltiples adapters
\end{itemize}

\column{0.4\textwidth}
\begin{figure}
    \centering
    %\includegraphics[width=0.9\textwidth]{Slides/figures/02_Metodos_Generativos/adapter-diagram.png}
    \caption{Adapter entre capas}
\end{figure}

\begin{block}{Comparación}
\small
\textbf{LoRA}: modifica pesos existentes

\textbf{Adapters}: añade nuevas capas
\end{block}

\end{columns}

\end{frame}

\begin{frame}{Prompt Tuning / Prefix Tuning}

\begin{columns}
\column{0.6\textwidth}
\textbf{Concepto}:
\begin{itemize}
    \item En lugar de entrenar pesos del modelo...
    \item Entrenar \textbf{embeddings continuos} que se añaden al input
    \item Modelo completamente congelado
\end{itemize}

\vspace{0.5em}

\textbf{Tipos}:
\begin{itemize}
    \item \textbf{Prompt Tuning}: soft prompts al inicio
    \item \textbf{Prefix Tuning}: vectores en cada capa
    \item \textbf{P-Tuning v2}: más sofisticado
\end{itemize}

\vspace{0.5em}

\textbf{Extremadamente eficiente}:
\begin{itemize}
    \item $<0.1\%$ de parámetros entrenables
    \item Ideal para modelos muy grandes
\end{itemize}

\column{0.4\textwidth}
\begin{exampleblock}{Ejemplo}
\small
Input original:
\texttt{Classify: [text]}

\vspace{0.3em}
Con prompt tuning:
\texttt{[P1][P2][P3] Classify: [text]}

donde $[P_i]$ son vectores entrenables
\end{exampleblock}

\vspace{0.5em}

\begin{alertblock}{Nota}
\small
Diferentes de los "prompts" que escribimos. Estos son vectores continuos, no texto.
\end{alertblock}

\end{columns}

\end{frame}

\begin{frame}{Comparación de Métodos PEFT}

\begin{table}[]
\centering
\small
\begin{tabular}{|l|c|c|c|c|}
\hline
\textbf{Método} & \textbf{Params} & \textbf{Rendimiento} & \textbf{Velocidad} & \textbf{Memoria} \\ \hline
Full Fine-tuning & 100\% & ★★★★★ & ★★ & ★★ \\ \hline
LoRA & 0.1-1\% & ★★★★ & ★★★★ & ★★★★★ \\ \hline
Adapters & 2-5\% & ★★★★ & ★★★ & ★★★★ \\ \hline
Prompt Tuning & <0.1\% & ★★★ & ★★★★★ & ★★★★★ \\ \hline
\end{tabular}
\end{table}

\vspace{1em}

\begin{block}{Recomendación práctica}
\begin{itemize}
    \item \textbf{Full fine-tuning}: si tienes recursos y datos abundantes
    \item \textbf{LoRA}: mejor balance general (nuestra recomendación)
    \item \textbf{Adapters}: si necesitas múltiples tareas
    \item \textbf{Prompt Tuning}: para modelos gigantes (>100B params)
\end{itemize}
\end{block}

\end{frame}

\begin{frame}{Casos de Uso: Fine-tuning en la Práctica}

\textbf{1. Clasificación específica de dominio}
\begin{itemize}
    \item Ejemplo: análisis de sentimiento de reviews médicas
    \item Base: BERT o RoBERTa
    \item Método: LoRA con $r=16$
\end{itemize}

\vspace{0.5em}

\textbf{2. Question Answering en documentos corporativos}
\begin{itemize}
    \item Base: T5 o BERT (QA)
    \item Datos: preguntas-respuestas del dominio
    \item Método: Full fine-tuning o LoRA
\end{itemize}

\vspace{0.5em}

\textbf{3. Resumen de artículos científicos}
\begin{itemize}
    \item Base: BART o T5
    \item Datos: papers + abstracts
    \item Método: LoRA (memoria limitada)
\end{itemize}

\vspace{0.5em}

\begin{alertblock}{Consideración importante}
Siempre evaluar en conjunto de test para evitar overfitting
\end{alertblock}

\end{frame}

\section{Optimización y Eficiencia}

\begin{frame}{Knowledge Distillation: Teacher-Student}

\begin{columns}
\column{0.6\textwidth}
\textbf{Objetivo}: Transferir conocimiento de un modelo grande (teacher) a uno pequeño (student)

\vspace{0.5em}

\textbf{Proceso}:
\begin{enumerate}
    \item Entrenar modelo grande (teacher)
    \item Usar outputs del teacher como "soft targets"
    \item Entrenar modelo pequeño (student) para imitarlo
\end{enumerate}

\vspace{0.5em}

\textbf{¿Por qué funciona?}
\begin{itemize}
    \item Las probabilidades suaves contienen más información
    \item El student aprende relaciones entre clases
    \item No solo la clase correcta, sino las similitudes
\end{itemize}

\column{0.4\textwidth}
\begin{figure}
    \centering
    %\includegraphics[width=\textwidth]{Slides/figures/02_Metodos_Generativos/distillation-diagram.png}
    \caption{Knowledge Distillation}
\end{figure}

\begin{exampleblock}{Caso de éxito}
\small
\textbf{DistilBERT}
\begin{itemize}
    \item 40\% menos parámetros
    \item 60\% más rápido
    \item 97\% del rendimiento de BERT
\end{itemize}
\end{exampleblock}

\end{columns}

\end{frame}

\begin{frame}{Knowledge Distillation: Función de Pérdida}

\textbf{Pérdida combinada}:
\begin{align*}
\mathcal{L} = \alpha \cdot \mathcal{L}_{CE}(\text{hard targets}) + (1-\alpha) \cdot \mathcal{L}_{KL}(\text{soft targets})
\end{align*}

donde:
\begin{itemize}
    \item $\mathcal{L}_{CE}$: cross-entropy con etiquetas reales
    \item $\mathcal{L}_{KL}$: KL-divergence con distribución del teacher
    \item $\alpha$: balance entre ambas pérdidas
\end{itemize}

\vspace{0.5em}

\textbf{Soft targets con temperatura}:
\begin{align*}
p_i = \frac{\exp(z_i / T)}{\sum_j \exp(z_j / T)}
\end{align*}

donde $T$ (temperatura) suaviza la distribución

\vspace{0.5em}

\begin{block}{Intuición}
$T > 1$ → distribución más suave → más información sobre similitudes entre clases
\end{block}

\end{frame}

\begin{frame}{Quantization: Reduciendo Precisión}

\begin{columns}
\column{0.5\textwidth}
\textbf{Idea}: Representar pesos con menos bits

\vspace{0.5em}

\textbf{Tipos de precisión}:
\begin{itemize}
    \item \textbf{FP32}: 32 bits (full precision)
    \item \textbf{FP16}: 16 bits (half precision)
    \item \textbf{INT8}: 8 bits (integer)
    \item \textbf{INT4}: 4 bits (muy agresivo)
\end{itemize}

\vspace{0.5em}

\textbf{Beneficios}:
\begin{itemize}
    \item Menos memoria (4x con INT8)
    \item Inferencia más rápida (2-4x)
    \item Importante para deployment
\end{itemize}

\column{0.5\textwidth}
\begin{table}[]
\centering
\small
\begin{tabular}{|l|c|c|}
\hline
\textbf{Tipo} & \textbf{Memoria} & \textbf{Velocidad} \\ \hline
FP32 & 1x & 1x \\ \hline
FP16 & 0.5x & 1.5-2x \\ \hline
INT8 & 0.25x & 2-4x \\ \hline
INT4 & 0.125x & 3-6x \\ \hline
\end{tabular}
\end{table}

\vspace{0.5em}

\begin{alertblock}{Trade-off}
Menor precisión → posible pérdida de calidad

En la práctica: INT8 funciona muy bien
\end{alertblock}

\end{columns}

\end{frame}

\begin{frame}{QLoRA: Quantization + LoRA}

\begin{columns}
\column{0.6\textwidth}
\textbf{Innovación}: Combinar lo mejor de ambos mundos

\vspace{0.5em}

\textbf{Cómo funciona}:
\begin{enumerate}
    \item Cargar modelo base en \textbf{4-bit} (quantizado)
    \item Añadir \textbf{adaptadores LoRA} en precisión normal
    \item Fine-tunear solo los adaptadores
    \item Resultado: modelo cuantizado + adaptadores entrenados
\end{enumerate}

\vspace{0.5em}

\textbf{Ventajas}:
\begin{itemize}
    \item Memoria mínima durante entrenamiento
    \item Permite fine-tunear modelos gigantes (65B+) en una GPU
    \item Rendimiento competitivo con full fine-tuning
\end{itemize}

\column{0.4\textwidth}
\begin{exampleblock}{Ejemplo impactante}
\small
\textbf{LLaMA-65B}

Sin QLoRA:
\begin{itemize}
    \item 130 GB VRAM
    \item Imposible en 1 GPU
\end{itemize}

Con QLoRA:
\begin{itemize}
    \item 48 GB VRAM
    \item Cabe en A100 (80GB)
\end{itemize}
\end{exampleblock}

\vspace{0.5em}

\begin{block}{Paper clave}
\small
"QLoRA: Efficient Finetuning of Quantized LLMs" (2023)
\end{block}

\end{columns}

\end{frame}

\begin{frame}{Otras Técnicas de Optimización}

\textbf{1. Pruning (Poda)}
\begin{itemize}
    \item Eliminar conexiones/neuronas poco importantes
    \item Structured vs. Unstructured pruning
    \item Puede reducir hasta 90\% de parámetros con pérdida mínima
\end{itemize}

\vspace{0.5em}

\textbf{2. Weight Sharing}
\begin{itemize}
    \item Compartir pesos entre capas (ej: ALBERT)
    \item Reduce parámetros sin reducir profundidad
\end{itemize}

\vspace{0.5em}

\textbf{3. Mixed Precision Training}
\begin{itemize}
    \item FP16 para forward/backward pass
    \item FP32 para actualización de pesos
    \item 2x más rápido, misma precisión
\end{itemize}

\vspace{0.5em}

\textbf{4. Gradient Checkpointing}
\begin{itemize}
    \item Trade-off: memoria ↔ velocidad
    \item Recomputar activaciones en lugar de guardarlas
\end{itemize}

\end{frame}

\section{RLHF y Modelos Instructivos}

\begin{frame}{El Problema: Modelos Pre-entrenados vs. Útiles}

\begin{columns}
\column{0.5\textwidth}
\textbf{Modelo pre-entrenado}:
\begin{itemize}
    \item Predice siguiente token
    \item No optimizado para seguir instrucciones
    \item Puede generar contenido inapropiado
    \item No alineado con valores humanos
\end{itemize}

\vspace{0.5em}

\textbf{Lo que queremos}:
\begin{itemize}
    \item Seguir instrucciones
    \item Ser útil
    \item Ser honesto
    \item Ser seguro (harmless)
\end{itemize}

\column{0.5\textwidth}
\begin{exampleblock}{Ejemplo}
\small
\textbf{Prompt}: "¿Cómo hago una tortilla?"

\vspace{0.3em}
\textbf{GPT-3 base}:
"¿Cómo hago una tortilla? ¿Cómo hago un pastel? ¿Cómo..."

\vspace{0.3em}
\textbf{ChatGPT (con RLHF)}:
"Para hacer una tortilla necesitas: huevos, aceite, sal..."
\end{exampleblock}

\end{columns}

\vspace{1em}

\begin{alertblock}{Solución}
\textbf{RLHF}: Reinforcement Learning from Human Feedback
\end{alertblock}

\end{frame}

\begin{frame}{RLHF: Las Cuatro Fases}

\begin{enumerate}
    \item \textbf{Pretraining} (modelo base)
    \begin{itemize}
        \item Entrenamiento estándar con next-token prediction
        \item Ejemplo: GPT-3, LLaMA base
    \end{itemize}
    
    \vspace{0.3em}
    
    \item \textbf{Supervised Fine-Tuning (SFT)}
    \begin{itemize}
        \item Humanos escriben ejemplos de alta calidad
        \item Fine-tunear el modelo con estos ejemplos
        \item Típicamente 10K-100K ejemplos
    \end{itemize}
    
    \vspace{0.3em}
    
    \item \textbf{Reward Modeling (RM)}
    \begin{itemize}
        \item Modelo genera múltiples respuestas
        \item Humanos las rankean (mejor → peor)
        \item Entrenar modelo de recompensa que predice rankings
    \end{itemize}
    
    \vspace{0.3em}
    
    \item \textbf{Reinforcement Learning (PPO)}
    \begin{itemize}
        \item Optimizar con Proximal Policy Optimization
        \item Maximizar recompensa según modelo RM
        \item KL-penalty para no alejarse demasiado del SFT
    \end{itemize}
\end{enumerate}

\end{frame}

\begin{frame}{RLHF: Diagrama del Proceso}

\begin{figure}
    \centering
    %\includegraphics[width=0.95\textwidth]{Slides/figures/02_Metodos_Generativos/rlhf-diagram.png}
    \caption{Pipeline completo de RLHF (Fuente: OpenAI)}
\end{figure}

\begin{block}{Modelos que usan RLHF}
ChatGPT, GPT-4, Claude, Llama 2 Chat, Mistral Instruct
\end{block}

\end{frame}

\begin{frame}{RLHF: Por qué es Importante}

\textbf{Impacto en la práctica}:
\begin{itemize}
    \item ChatGPT → explosión de uso de LLMs
    \item Modelos más seguros y útiles
    \item Mejor seguimiento de instrucciones
    \item Reducción de outputs tóxicos
\end{itemize}

\vspace{0.5em}

\textbf{Comparación} (GPT-3 vs. GPT-3.5-turbo):
\begin{itemize}
    \item GPT-3: modelo base
    \item GPT-3.5-turbo: + SFT + RLHF
    \item Diferencia dramática en utilidad
\end{itemize}

\vspace{0.5em}

\begin{exampleblock}{Estadística}
OpenAI reporta: RLHF mejora alineación en $>2x$ comparado con solo SFT
\end{exampleblock}

\end{frame}

\begin{frame}{RLHF: Limitaciones y Desafíos}

\textbf{1. Coste de Anotación}
\begin{itemize}
    \item Necesita muchas comparaciones humanas
    \item Anotadores especializados (costoso)
    \item Difícil de escalar
\end{itemize}

\vspace{0.5em}

\textbf{2. Reward Hacking}
\begin{itemize}
    \item El modelo aprende a "engañar" al reward model
    \item Respuestas que parecen buenas pero son incorrectas
    \item Sycophancy: decirle al usuario lo que quiere oír
\end{itemize}

\vspace{0.5em}

\textbf{3. Sesgos y Valores}
\begin{itemize}
    \item ¿Qué valores codificamos?
    \item Diferentes culturas, diferentes preferencias
    \item Riesgo de amplificar sesgos de anotadores
\end{itemize}

\vspace{0.5em}

\textbf{4. Alternativas en Desarrollo}
\begin{itemize}
    \item Constitutional AI (Anthropic)
    \item RLAIF: RL from AI Feedback (usar LLMs para generar feedback)
\end{itemize}

\end{frame}

\section{RAG y Agentes}

\begin{frame}{El Problema del Conocimiento Estático}

\begin{columns}
\column{0.5\textwidth}
\textbf{Limitaciones de los LLMs}:
\begin{itemize}
    \item Conocimiento "congelado" en el entrenamiento
    \item No saben eventos recientes
    \item No tienen acceso a info privada/corporativa
    \item Pueden "alucinar" información
\end{itemize}

\vspace{0.5em}

\textbf{Ejemplo}:
\begin{itemize}
    \item GPT-4 (cutoff: sep 2021)
    \item Pregunta: "¿Quién ganó el Mundial 2022?"
    \item Respuesta: no lo sabe
\end{itemize}

\column{0.5\textwidth}
\begin{alertblock}{Solución}
\textbf{RAG}: Retrieval-Augmented Generation

\vspace{0.3em}
Darle al modelo acceso a información externa
\end{alertblock}

\vspace{0.5em}

\begin{exampleblock}{Casos de uso}
\begin{itemize}
    \item QA sobre documentación
    \item Chat con PDFs
    \item Búsqueda semántica
    \item Knowledge bases corporativas
\end{itemize}
\end{exampleblock}

\end{columns}

\end{frame}

\begin{frame}{RAG: Arquitectura}

\begin{figure}
    \centering
%    \includegraphics[width=0.9\textwidth]{Slides/figures/02_Metodos_Generativos/rag-architecture.png}
    \caption{Arquitectura típica de RAG}
\end{figure}

\textbf{Componentes}:
\begin{enumerate}
    \item \textbf{Knowledge Base}: documentos, artículos, base de datos
    \item \textbf{Retriever}: busca documentos relevantes (encoder, sentence-transformers)
    \item \textbf{Generator}: genera respuesta usando contexto (LLM, T5, GPT)
\end{enumerate}

\end{frame}

\begin{frame}{RAG: ¿Cómo Funciona?}

\textbf{Pipeline paso a paso}:

\vspace{0.5em}

\begin{enumerate}
    \item \textbf{Indexación} (offline):
    \begin{itemize}
        \item Procesar documentos
        \item Generar embeddings (ej: sentence-transformers)
        \item Guardar en vector database (FAISS, Pinecone, Weaviate)
    \end{itemize}
    
    \vspace{0.3em}
    
    \item \textbf{Retrieval} (online):
    \begin{itemize}
        \item Usuario hace pregunta
        \item Convertir pregunta a embedding
        \item Buscar documentos más similares (cosine similarity, etc.)
    \end{itemize}
    
    \vspace{0.3em}
    
    \item \textbf{Generation} (online):
    \begin{itemize}
        \item Construir prompt: pregunta + contexto recuperado
        \item LLM genera respuesta basada en contexto
    \end{itemize}
\end{enumerate}

\vspace{0.5em}

\begin{block}{Clave}
El modelo no "memoriza" información, la \textbf{busca y usa} cuando la necesita
\end{block}

\end{frame}

\begin{frame}[fragile]{RAG: Ejemplo de Prompt}

\begin{block}{Prompt construido por RAG}
\small
\begin{verbatim}
Use the following context to answer the question.

Context:
[Documento 1]: Los transformers fueron introducidos 
en 2017 en el paper "Attention is All You Need"...

[Documento 2]: BERT es un modelo encoder-only que 
usa masked language modeling...

Question: ¿Cuándo se inventaron los transformers?

Answer:
\end{verbatim}
\end{block}

\vspace{0.5em}

\textbf{Ventajas}:
\begin{itemize}
    \item Respuestas basadas en fuentes verificables
    \item Actualización fácil (actualizar base de datos, no reentrenar)
    \item Reduce alucinaciones
\end{itemize}

\end{frame}

\begin{frame}{RAG: Consideraciones Técnicas}

\textbf{Retriever}:
\begin{itemize}
    \item \textbf{Dense retrieval}: embeddings (BERT, sentence-transformers)
    \item \textbf{Sparse retrieval}: BM25, TF-IDF
    \item \textbf{Hybrid}: combinar ambos
\end{itemize}

\vspace{0.5em}

\textbf{Chunking Strategy}:
\begin{itemize}
    \item ¿Cuánto texto por chunk? (típicamente 200-500 tokens)
    \item ¿Overlapping? (mejor capturar contexto)
\end{itemize}

\vspace{0.5em}

\textbf{Vector Databases}:
\begin{itemize}
    \item FAISS (Facebook): in-memory, muy rápido
    \item Pinecone, Weaviate, Milvus: managed, escalables
    \item ChromaDB: simple, open-source
\end{itemize}

\vspace{0.5em}

\textbf{Trade-offs}:
\begin{itemize}
    \item Más documentos recuperados → más contexto pero más coste
    \item Calidad de retrieval crítica para calidad de respuesta
\end{itemize}

\end{frame}

\begin{frame}{Agentes LLM: Introducción}

\begin{columns}
\column{0.6\textwidth}
\textbf{¿Qué es un agente?}
\begin{itemize}
    \item LLM como "cerebro" que decide acciones
    \item Capacidad de usar \textbf{herramientas externas}
    \item Loop de razonamiento → acción → observación
\end{itemize}

\vspace{0.5em}

\textbf{Herramientas típicas}:
\begin{itemize}
    \item Calculadora
    \item Búsqueda web (Google, Wikipedia)
    \item APIs (weather, stock prices, etc.)
    \item Ejecutar código (Python REPL)
    \item Acceso a bases de datos
\end{itemize}

\vspace{0.5em}

\textbf{Arquitecturas}:
\begin{itemize}
    \item \textbf{ReAct}: Reasoning + Acting
    \item \textbf{Plan-and-Execute}: planificar primero, luego ejecutar
\end{itemize}

\column{0.4\textwidth}
\begin{exampleblock}{Ejemplo}
\small
\textbf{Pregunta}: "¿Cuál es la raíz cuadrada de 1764?"

\vspace{0.3em}
\textbf{Agente}:
\begin{enumerate}
    \item Reconoce que necesita cálculo
    \item Usa herramienta "calculadora"
    \item Obtiene resultado: 42
    \item Responde al usuario
\end{enumerate}
\end{exampleblock}

\end{columns}

\end{frame}

\begin{frame}{Agentes: ReAct Pattern}

\textbf{ReAct}: Reasoning + Acting (interleaved)

\vspace{0.5em}

\begin{block}{Ejemplo de trace}
\small
\textbf{Question}: What is the elevation of the highest point in California?

\vspace{0.3em}
\textbf{Thought 1}: I need to search for the highest point in California

\textbf{Action 1}: Search[highest point in California]

\textbf{Observation 1}: Mount Whitney is the highest point...

\vspace{0.3em}
\textbf{Thought 2}: Now I need to find its elevation

\textbf{Action 2}: Search[Mount Whitney elevation]

\textbf{Observation 2}: 4,421 meters

\vspace{0.3em}
\textbf{Thought 3}: I have the answer

\textbf{Action 3}: Finish[4,421 meters]
\end{block}

\end{frame}

\begin{frame}{Frameworks para Agentes}

\textbf{1. LangChain}
\begin{itemize}
    \item Framework más popular
    \item Abstrae agents, chains, tools
    \item Gran ecosistema de integraciones
\end{itemize}

\vspace{0.5em}

\textbf{2. LlamaIndex}
\begin{itemize}
    \item Especializado en RAG y data connectors
    \item Indexación y query de documentos
    \item Múltiples estrategias de retrieval
\end{itemize}

\vspace{0.5em}

\textbf{3. Otros}
\begin{itemize}
    \item AutoGPT, BabyAGI: agentes autónomos
    \item Semantic Kernel (Microsoft)
    \item Haystack
\end{itemize}

\vspace{0.5em}

\begin{alertblock}{Nota}
Campo muy activo, nuevas herramientas cada mes
\end{alertblock}

\end{frame}

\begin{frame}{Agentes: Limitaciones}

\textbf{Desafíos actuales}:
\begin{itemize}
    \item \textbf{Confiabilidad}: pueden fallar o entrar en loops
    \item \textbf{Coste}: muchas llamadas a LLM ($\$$)
    \item \textbf{Latencia}: múltiples pasos → respuesta lenta
    \item \textbf{Evaluación}: difícil medir rendimiento
    \item \textbf{Seguridad}: ejecución de código, acceso a APIs
\end{itemize}

\vspace{0.5em}

\textbf{Buenas prácticas}:
\begin{itemize}
    \item Limitar número de pasos
    \item Validar acciones antes de ejecutar
    \item Logging detallado para debugging
    \item Empezar simple, añadir complejidad gradualmente
\end{itemize}

\vspace{0.5em}

\begin{block}{Estado actual (2024-2025)}
Prometedores para casos de uso específicos, pero aún no "production-ready" para casos generales
\end{block}

\end{frame}

\begin{exercise}
\href{https://colab.research.google.com/}{Práctica: Fine-tuning y Optimización}

En este notebook exploraremos:
\begin{itemize}
    \item Full fine-tuning (baseline)
    \item PEFT con LoRA
    \item Quantization (INT8)
    \item QLoRA (bonus)
    \item Mini RAG (conceptual)
\end{itemize}

Compararemos: parámetros entrenables, tiempo, memoria, y accuracy.
\end{exercise}

%%%%%%%%%%%%%%%%%%%%%%%%%%%%

\begin{frame}{Recursos}
\begin{itemize}
    \item PEFT Library: \url{https://github.com/huggingface/peft}
    \item LoRA Paper: \url{https://arxiv.org/abs/2106.09685}
    \item QLoRA Paper: \url{https://arxiv.org/abs/2305.14314}
    \item RLHF Blog (OpenAI): \url{https://openai.com/research/learning-from-human-preferences}
    \item LangChain: \url{https://www.langchain.com/}
    \item LlamaIndex: \url{https://www.llamaindex.ai/}
    \item RAG Survey: \url{https://arxiv.org/abs/2312.10997}
\end{itemize}
\end{frame}

\appendix

\begin{frame}<presentation:0>{License}
    \begin{block}{Tema \texttt{slides-upm}. Puedes obtener sus fuentes en}
        \begin{center}\url{http://gitlab.com/blazaid/slides-upm}\end{center}
    \end{block}
  
    Tanto esta presentación como el tema están licenciados bajo \href{http://creativecommons.org/licenses/by-sa/4.0/}{Creative Commons
  Atribución-CompartirIgual 4.0 Internacional (CC BY-SA 4.0)}.
    \begin{center}\ccbysa\end{center}
\end{frame}

\end{document}
